\begin{longtabu} to \textwidth {|r|X[6,l]|X[3,l]|X[2,l]|}
\hline
P.        &Context                                                                                             &Suggestion                              &Reason              \\
\hhline{|=|=|=|=|}
         1&Introduction \genbf{And} Definitions                                                                &and                                     &readability         \\\cline{2-4}
          &and so is expression $42 \genbf{*} 2$                                                               &use TeX cdot                            &readability         \\\cline{2-4}
\hhline{|=|=|=|=|}
         2&\genbf{and we} use capitalized names for constructors                                               &We                                      &readability         \\\cline{2-4}
          &starting on \genbf{data further from a base case}                                                   &data from a special case                &readability         \\\cline{2-4}
          &for \genbf{an} SMT solver                                                                           &a                                       &grammar             \\\cline{2-4}
          &We use \textit{many-sorted} first-order logic, that is, \genbf{logic} with                          &a logic                                 &grammar             \\\cline{2-4}
\hhline{|=|=|=|=|}
         3&A type $\delta$ depends on another type \genbf{$\epsilon$ is $\epsilon$ is} the type of             &$\epsilon$ is                           &repetition          \\\cline{2-4}
          &We also need \genbf{auxilitary} functions                                                           &auxiliary                               &typo                \\\cline{2-4}
          &Distinctness:~\dots~\genbf{no two constructors} are equal                                           &no two constructors of the same type    &readability         \\\cline{2-4}
          &\genbf{We then define theory of datatypes and codatatypes $\dct$}                                   &We then define $\dct$ -- the theory \dots&readability         \\\cline{2-4}
          &It defines a class of \genbf{$\Sigma$-interpretations~~which}                                       &possibly the symbol for this class is missing&notation            \\\cline{2-4}
          &A formula is called \textit{DC-satisfiable} if there exists an interpretation \genbf{in that} satisfies it&something is missing                    &readability         \\\cline{2-4}
          &An instance of degenerated codatatype~\dots~with \genbf{an} unique value                            &a                                       &grammar             \\\cline{2-4}
\hhline{|=|=|=|=|}
         4&Lemma 1.~\dots~a singleton \genbf{(cardinality equals 1)}                                           &with cardinality equal to 1             &readability         \\\cline{2-4}
          &In the latter case, $\delta$ necessarily \genbf{as} a single constructor                            &has                                     &typo                \\\cline{2-4}
          &$C(1, C(0, C(0, \dots)))$, $C(0, C(1, C(0, \dots)))$ \genbf{etc}                                    &etc.                                    &grammar             \\\cline{2-4}
          &The procedure \genbf{builts} \genbf{closure} of $E$ under given rules which are assigned a \genbf{phase; rules} belonging \dots&builds; a closure; phase. Rules         &grammar             \\\cline{2-4}
          &where \genbf{conclusion} either specifies                                                           &the conclusion                          &grammar             \\\cline{2-4}
          &or is $\bot$ (\genbf{contradiction})                                                                &a contradiction                         &grammar             \\\cline{2-4}
          &and \genbf{children} of a node are \genbf{results} of rule application                              &the children; the results               &grammar             \\\cline{2-4}
          &one by one \genbf{in spirit} of depth-first search                                                  &in a spirit                             &grammar             \\\cline{2-4}
          &with leaf nodes \genbf{$\bot$ allowed}                                                              &need to rephrase ``contradiction allowed''&readability         \\\cline{2-4}
          &A \genbf{note} is saturated if no nonredundant rule \genbf{application} can be performed            &node; applications                      &grammar             \\\cline{2-4}
          &a calculus of equality \genbf{sets} with applications as operations                                 &of sets                                 &grammar             \\\cline{2-4}
          &$\mathrm{Refl}$, $\mathrm{Sym}$ and $\mathrm{Trans}$, \genbf{Refl} are encoded properties           &-                                       &repetition          \\\cline{2-4}
          &whereas $\mathrm{Inject}$ computes \genbf{unification} (downward) closure                           &the unification                         &grammar             \\\cline{2-4}
\hhline{|=|=|=|=|}
         5&We write $\mapping(x)$ to get a representative for \genbf{equivalence class} of $x$                 &an/the equivalence class                &grammar             \\\cline{2-4}
          &$\mathrm{Clash}$ represents \genbf{failure} to unify                                                &a/the failure                           &grammar             \\\cline{2-4}
          &Figure 1. The $\mathrm{Clash}$ rule: $\genbf{C}(t_1,\dots,t_n) \approx \genbf{\mathcal{D}}(u_1,\dots,u_n) \in E \quad \genbf{C} \neq \genbf{D}$&-; -; not unified display style for the same entities&notation            \\\cline{2-4}
          &$C \in~$\genbf{$\func{}$$F_{ctr}$}                                                                  &$\func{}_{ctr}$                         &typo                \\\cline{2-4}
          &All datatypes therefore~\dots~and all codatatypes -- as \genbf{cyclic}                              &ones                                    &grammar             \\\cline{2-4}
          &$\alpha$-disequivalent\genbf{---}e.g.                                                               &too long dash                           &typo                \\\cline{2-4}
\hhline{|=|=|=|=|}
         6&Hence, $\mapping(t^\tau)$~\dots~can take in models of \genbf{$E$. by} \genbf{appling}               &$E$ by; applying                        &grammar, punctuation\\\cline{2-4}
          &\genbf{Because of $\mu$-terms}, \genbf{there is} no infinitely expanding \genbf{terms so} \genbf{number} of times&Because of $\mu$-terms what?; \newline not stated where afterwards; \newline terms\genbf{,}~so; \newline the number&readability, grammar, punctuation\\\cline{2-4}
          &They capture \genbf{properties} of (co)datatypes described above                                    &the properties                          &grammar             \\\cline{2-4}
          &\genbf{Informally first rule} of this phase~\dots~implements \genbf{search}                         &Informally\genbf{,} the first rule; searching&grammar             \\\cline{2-4}
          &The second rule is needed to deal with \genbf{described} degenerate case                            &the described                           &grammar             \\\cline{2-4}
          &with \genbf{root node}                                                                              &the root node (multiple occurences found throughout the paper)&grammar             \\\cline{2-4}
          &If $\mathrm{Split}$ is applied on some $t^\tau$, \genbf{by} the previous step                       &at                                      &grammar             \\\cline{2-4}
\hhline{|=|=|=|=|}
         7&Now we construct \genbf{family} of sets                                                             &a family                                &grammar             \\\cline{2-4}
          &Consider $S_t^\infty$~\dots~because all chains of selectors \genbf{in input}                        &its/the input                           &grammar             \\\cline{2-4}
          &\genbf{Depth} of a tree branch is therefore                                                         &The depth                               &grammar             \\\cline{2-4}
          &First we define \textit{\genbf{expansion}}~\dots~with \genbf{free variable} $x$ \genbf{is} substituted by \genbf{body} of $t$&the expansion; \newline a free variable; \newline unnecessary ``is''; \newline the body&grammar             \\\cline{2-4}
          &By definition of \genbf{$\mu$-terms after}                                                          &$\mu$-terms\genbf{,} after              &punctuation         \\\cline{2-4}
\hhline{|=|=|=|=|}
         8&Finally a normal form~\dots~recursively normalizing \genbf{the other} subterms                      &other                                   &grammar             \\\cline{2-4}
          &\genbf{Now that we defined}                                                                         &Having defined/\newline Now that we have defined&grammar, readability\\\cline{2-4}
          &$\interp_\type(\tau)$ denotes \genbf{domain} of type $\tau$                                         &the domain                              &grammar             \\\cline{2-4}
          &terms that are not \genbf{equalized}                                                                &equal/unified                           &readability         \\\cline{2-4}
          &We will base the process on \genbf{mapping} $\mapping$ from section                                 &the mapping                             &grammar             \\\cline{2-4}
\hhline{|=|=|=|=|}
         9&The set is constructed~\dots~by exhaustively \genbf{appling}                                        &applying                                &typo                \\\cline{2-4}
          &\genbf{First} point can be proven                                                                   &The first                               &grammar             \\\cline{2-4}
          &Points 2 and 3 can be proven by induction. \genbf{First one needs to prove that}                    &To prove/for proving the first one, we need to prove \dots&grammar, readability\\\cline{2-4}
          &\genbf{Next one} needs to prove                                                                     &Next\genbf{,} one                       &grammar             \\\cline{2-4}
\hhline{|=|=|=|=|}
        10&Then because $\mapping$ is an equivalence class \genbf{that implies} all equalities~\dots~are satisfied&The whole sentence needs to be rephrased to mitigate the ambiguity of its interpretation. E.g.: ``Then, taking into account, that $\mapping$ is an equivalence class, all qualities \dots''&readability         \\\cline{2-4}
          &by induction hypothesis \genbf{then} it's                                                           &an unnecessary word                     &readability         \\\cline{2-4}
          &by induction hypothesis then \genbf{it's}                                                           &it is                                   &informal language   \\\cline{2-4}
          &\genbf{We use the fact that because} we consider a saturated node                                   &As a saturated node is considered, \dots&readability         \\\cline{2-4}
          &If we let $u_i'$ be \genbf{$i$th} argument                                                          &the $i$th                               &grammar             \\\cline{2-4}
          &We \genbf{can't} apply $\mathrm{Split}$                                                             &cannot                                  &informal language   \\\cline{2-4}
          &If $j \neq j'$ \genbf{we use that} $\mathrm{Cong}$ is exhausted                                     &not unnecessary words                   &readability         \\\cline{2-4}
          &$\mathrm{Cong}$ is \genbf{exhausted so} for any                                                     &exhausted\genbf{,} so                   &punctuation         \\\cline{2-4}
          &$C_{j'}\left( s_{j'}^1(u),\dots,s_{j'}^n(u) \right)$ \genbf{for unknown yet} $j'$                   &for yet unknown                         &grammar             \\\cline{2-4}
          &$\mathrm{Conflict}$ does not \genbf{apply so} \dots                                                 &apply\genbf{,} so                       &punctuation         \\\cline{2-4}
          &Because $\interp_\func(t) =_\alpha \mapping^*(t)$~\dots~and they are not equal by \genbf{lemma 4(3) we} have&lemma 4(3)\genbf{,} we                  &punctuation         \\\cline{2-4}
\hhline{|=|=|=|=|}
\end{longtabu}
